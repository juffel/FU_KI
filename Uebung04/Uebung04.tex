\documentclass[10pt,a4paper]{article}
\usepackage[utf8]{inputenc}
\usepackage{amsmath}
\usepackage{amsfonts}
\usepackage{amssymb}
\usepackage{graphicx}
\begin{document}
\section*{Aufgabe 1}
	\subsection*{a) $\forall Z \exists Y \forall X (f(X, Y))\Leftrightarrow (f(X, Z) \wedge \neg f(X, X)) $}
	Äquivalenz in zwei Implikationen umformen:
	\begin{gather*}
		\forall Z \exists Y \forall X ((f(X, Y) \Rightarrow (f(X,Z) \wedge \neg f(X, X))) \\ \wedge 
		(f(X,Z) \wedge \neg f(X, X)) \Rightarrow f(X, Y))
	\end{gather*}
	Implikationen in Disjunktionen umformen:
	\begin{gather*}
		\forall Z \exists Y \forall X (\neg f(X, Y) \vee (f(X,Z) \wedge \neg f(X, X))) \\ 
		\wedge 
		\neg (f(X,Z) \wedge \neg f(X, X)) \vee f(X, Y))
	\end{gather*}
	Negation in der zweiten Zeile auflösen:
	\begin{gather*}
		\forall Z \exists Y \forall X (\neg f(X, Y) \vee (f(X,Z)
		\wedge \neg f(X, X)) 
		\\ \wedge 
		(\neg f(X,Z) \vee f(X, X) \vee f(X, Y)))
	\end{gather*}	
	erste Zeile aufteilen in Konjunktion von zwei Disjunktionen:
	\begin{gather*}
		\forall Z \exists Y \forall X 
		((\neg f(X, Y) \vee (f(X,Z)) \\
		\wedge (\neg f(X, Y) \vee \neg f(X, X))) \\  
		\wedge 
		(\neg f(X,Z) \vee f(X, X) \vee f(X, Y)))
	\end{gather*}
	Skolemisieren, so dass $\exists Y \equiv skY(Z)$:
	\begin{gather*}
		(\neg f(X, skY(Z)) \vee f(X,Z)) \\
		\wedge (\neg f(X, skY(Z)) \vee \neg f(X, X)) \\  
		\wedge 
		(\neg f(X,Z) \vee f(X, X) \vee f(X, skY(Z)))
	\end{gather*}
	schließlich bringen wir das ganze in die Mengenschreibweise der KNF:\\
	\{$\neg f(X, skY(Z)) \vee f(X,Z)$, $\neg f(X, skY(Z)) \vee \neg f(X, X)$, $\neg f(X,Z) \vee f(X, X) \vee f(X, skY(Z))$\}
	
	
	\subsection*{b)$ \forall X \forall Y (q(X, Y)) \Leftrightarrow \forall Z (f(Z, X) \Leftrightarrow f(Z, Y))$}
	aufteilen:
	\begin{gather*}
		\forall X \forall Y (q(X, Y)) \Leftrightarrow \forall Z (f(Z, X)) \\
		\forall Z (f(Z, X))	\Leftrightarrow f(Z, Y)
	\end{gather*}
	
	\subsection*{c) $\forall X \exists Y((p(X, Y)\Leftarrow \forall X \exists T q(Y, X, T))\Rightarrow r(Y)$} 
	aufteilen:
	\begin{gather*}
		\forall X \exists Y(\forall X \exists T q(Y, X, T))\Rightarrow r(Y) \\ 
		\wedge 
		(\forall X \exists T q(Y, X, T) \Rightarrow p(X, Y))
	\end{gather*}
	in Disjunktionen umwandeln:
	\begin{gather*}
		\forall X \exists Y(\neg(\forall X \exists T q(Y, X, T))) \vee r(Y) \\ 
		\wedge 
		(\neg(\forall X \exists T q(Y, X, T))\vee p(X, Y))
	\end{gather*}	
	Quantoren nach außen ziehen:
	\begin{gather*}
		\forall X \exists Y((\exists X \forall T (\neg q(Y, X, T) \vee r(Y)) \\ 
		\wedge 
		(\exists X \forall T (\neg q(Y, X, T)\vee p(X, Y)))
	\end{gather*}	
	
	d)
	
\section*{Aufgabe 3}

\subsection*{gegeben}
$$V=\{X,Y\}$$
$$F=\{vater\_von/1, mutter\_von/, max/0\}$$
$$P=\{verheiratet/2\}$$

\subsection*{gesucht}
\subsubsection*{Herbrand Universum}
\begin{align*}
max, \\ vater\_von(max), \\ vater\_von(vater\_von(max)) \\ ...\\
mutter\_von(max), \\ mutter\_von(mutter\_von(max)) \\ ... \\
vater\_von(mutter\_von(max)), \\
vater\_von(vater\_von(mutter\_von(max))), \\
mutter\_von(vater\_von(max)), \\
mutter\_von(vater\_von(vater\_von(max))), \\
... \\
mutter\_von(vater\_von(mutter\_von(vater\_von(mutter\_von(...)))))
\end{align*}

\subsubsection*{Herbrand Basis}
\begin{align*}
verheiratet(max, vater\_von(max)), \\
verheiratet(max, mutter\_von(max)), \\
verheiratet(vater\_von(max), mutter\_von(max)),\\
verheiratet(vater\_von(vater\_von(max)), mutter\_von(vater\_von(max))), \\
...
\end{align*}

\subsection*{Herbrand Interpretation}
\subsubsection*{D}
D ist das Herbrand Universum

\subsubsection*{F}
Die Identitätsfunktion:
\begin{align*}
max \rightarrow max, \\
vater\_von(max) \rightarrow vater\_von(max), \\
...
\end{align*}
\subsubsection*{R}
weist den Elementen der Herbrand Basis Wahrheitswerte zu:
\begin{align*}
verheiratet(max, vater\_von(max)) \rightarrow FALSE, \\
verheiratet(vater\_von(max), mutter\_von(max)) \rightarrow TRUE, \\
verheiratet(vater\_von(vater\_von(max)), mutter\_von(vater\_von(max))) \rightarrow TRUE, \\
...
\end{align*}

\end{document}	

